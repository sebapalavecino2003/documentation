\subsection*{Objetivo}
Generar el PDF (\texttt{main.pdf}) en tu computadora y dejar listo el flujo para trabajar junto con Overleaf si hace falta.

\subsection*{Requisitos (Windows)}
\begin{itemize}
  \item MiKTeX instalado (distribución \LaTeX{}).
  \item \textbf{Opción A (simple):} usar \texttt{pdflatex}.
  \item \textbf{Opción B (cómoda):} usar \texttt{latexmk} (requiere tener Perl instalado).
\end{itemize}

\subsection*{Compilar con \texttt{pdflatex} (opción simple)}
Desde la carpeta del repositorio:
\begin{verbatim}
pdflatex -interaction=nonstopmode -file-line-error main.tex
pdflatex -interaction=nonstopmode -file-line-error main.tex
\end{verbatim}
(Se corre dos veces para actualizar índice). El PDF queda como \texttt{main.pdf}.

\subsection*{Compilar con \texttt{latexmk} (opción cómoda)}
\textbf{Si falla por ``falta Perl''}, instalar Strawberry Perl y reabrir la terminal. Luego:
\begin{verbatim}
latexmk -pdf -interaction=nonstopmode -file-line-error main.tex
\end{verbatim}
Para limpiar archivos temporales:
\begin{verbatim}
latexmk -c
\end{verbatim}

\subsection*{Estructura mínima esperada}
\begin{verbatim}
main.tex
chapters/01-introduccion.tex
chapters/02-alcance.tex
chapters/03-github.tex
chapters/04-latex-local.tex
chapters/05-overleaf.tex
\end{verbatim}

\subsection*{Errores comunes (y solución rápida)}
\begin{itemize}
  \item \textbf{No se reconoce \texttt{pdflatex}}: MiKTeX no está en PATH. Reinstalar/abrir MiKTeX Console y reiniciar la terminal.
  \item \textbf{Faltan paquetes} (ventana de MiKTeX): aceptar instalar paquetes automáticamente.
  \item \textbf{\texttt{latexmk} falla}: instalar Perl y volver a correr (o usar \texttt{pdflatex} dos veces).
\end{itemize}

\subsection*{Checklist}
\begin{itemize}
  \item Compilar sin errores y abrir \texttt{main.pdf}.
  \item Hacer commit + push de los \textit{.tex} (no subir archivos temporales).
  \item (Opcional) Publicar el PDF en un Release de GitHub.
\end{itemize}
