Esta documentación tiene como objetivo centralizar, en un único lugar, toda la información
necesaria para crear, mantener y compartir el material del proyecto en \LaTeX. Está pensada
para un inicio rápido durante la pasantía: compilar el PDF en la computadora, versionar los
cambios en GitHub y (si corresponde) editar también en Overleaf.

\textbf{¿Qué vas a encontrar?}
\begin{itemize}
  \item Una estructura mínima de archivos (\texttt{main.tex}, \texttt{chapters/}, \texttt{figures/}, \texttt{bib/}).
  \item Pasos simples para compilar en Windows usando \texttt{latexmk}.
  \item Un flujo básico de trabajo con GitHub: ramas, commits, Pull Requests y Releases.
  \item Dos formas de sincronizar con Overleaf (URL de Git o proyecto creado desde GitHub).
\end{itemize}

\textbf{¿Por qué así?} Porque mantener la documentación en texto plano (\LaTeX) y con control
de versiones (GitHub) permite:
\begin{itemize}
  \item Historial claro de cambios (quién, cuándo y qué).
  \item Trabajo en equipo sin pisarse (ramas y revisiones).
  \item Reproducibilidad: cualquiera puede compilar el mismo PDF con los mismos pasos.
\end{itemize}

\textbf{A quién va dirigida.} A miembros del equipo que necesiten consultar o
actualizar la documentación, aun con experiencia limitada en \LaTeX{} y GitHub.

\textbf{Cómo usar este documento.} Leé la sección de GitHub para entender el flujo de trabajo,
seguí los pasos de \emph{LaTeX en local} para generar \texttt{main.pdf} y, si editás online,
revisá \emph{Sincronizar con Overleaf}. Cuando cierres una entrega, publicá un Release con el PDF.
